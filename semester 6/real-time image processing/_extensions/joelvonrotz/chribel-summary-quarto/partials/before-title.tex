
% [GENERAL PACKAGES] ------------------------------------------------------------------- %
% includes some neato math environments, symbols and stuff
\usepackage{amssymb, amsmath}

% correctly encodes the formatting of the text files (since we live in the future of UTF8)
\usepackage[utf8]{inputenc} 

% Used for the page counter in the footer to see the amount of pages 
\usepackage{lastpage} 

% Mostly used for the titlepage. Compared to the 'twocolumn' mode, multicols allows for
% more flexible changing (without inserting newpage between column mode changes)
\usepackage{multicol}
\setlength\columnsep{20pt}

% Package for enabling colors (colorful output)
\usepackage[dvipsnames]{xcolor}
\definecolor{colorAccent}{HTML}{$accentcolor$}

% Icons - http://mirrors.ctan.org/fonts/fontawesome5/doc/fontawesome5.pdf
% This one is neat, when you want to add some icons
\usepackage{fontawesome5}

% Depending on the type of document, mostly summaries, I remove the spacing Display-styled Equations insert, since it's a waste of space.
\usepackage[nodisplayskipstretch]{setspace}

% [FONT CONFIGURATION] ----------------------------------------------------------------- %
% change the fonts here (depending which latex engine is used, this might need to be
% removed)

$for(chribel-fontfamily)$
$if(it.options)$
\usepackage[$it.options$]{$it.name$}
$else$
\usepackage{$it.name$}
$endif$
$endfor$

% \usepackage{AlegreyaSans}
% \usepackage{cmbright}
% \usepackage[scaled=0.95]{inconsolata} % for code blocks

% [TIKZ CONFIGURATION] ----------------------------------------------------------------- %
% Tikz is really good at creating nice looking and clean drawings, diagramms, etc. 
\usepackage{tikz}
\usetikzlibrary{shapes,arrows,arrows.meta,matrix,decorations.pathmorphing}
\newcommand*\circled[1]{\tikz[baseline=(char.base)]{
            \node[shape=circle,draw,inner sep=2pt] (char) {#1};}}           

% TIP: use 'TikzEdt' (http://www.tikzedt.org/) to preview the drawings.

% [HEADING/SECTION STYLE CONFIGURATION] ------------------------------------------------ %
% 'sectsty' is used for minor adjustments and is used to apply the bold sans serif style
% to all sections (sec, subsec, subsubsec, ...)
\usepackage{sectsty}
\allsectionsfont{\normalfont\bfseries\sffamily}

% 'titlesec' is used to customize the numbered and unnumbered sections styles (but can
% also be used for the other section types).
% 'xhfill' is used to add a ruler, centered to the text.
\usepackage{xhfill}
\usepackage[explicit]{titlesec}

\makeatletter
\newcommand \Cdotfill {\leavevmode\cleaders\hb@xt@.25em{\hss$$\cdot$$\hss}\hfill\kern\z@}
\makeatother

% NUMBERED - SECTIION
\titleformat
  {\section} % {⟨command⟩}
  [hang] % [⟨shape⟩]
  {\normalfont\LARGE\bfseries\filright\sffamily} % {⟨format⟩}
  {\thesection} % {⟨label⟩}
  {3mm} % {⟨sep⟩}
  {%
    \color{colorAccent}{#1\hspace{2mm}\xrfill[0.6ex]{1pt}}
  } % {⟨before-code⟩}

\titlespacing*{\section} % ⟨command⟩
  {0em} % ⟨left⟩
  {12pt} % ⟨before-sep⟩
  {6pt} % ⟨after-sep⟩

% NUMBERLESS - SECTION
\titleformat
  {name=\section,numberless} % {⟨command⟩}
  [hang] % [⟨shape⟩]
  {\normalfont\LARGE\bfseries\filright\sffamily} % {⟨format⟩}
  {} % {⟨label⟩}
  {0mm} % {⟨sep⟩}
  {%
    \color{colorAccent}{#1\hspace{2mm}}\xrfill[0.6ex]{1.2pt}[colorAccent]
  }% {⟨before-code⟩}

% NUMBERED - SUBSECTION
\titleformat{\subsection} % {⟨command⟩}
  [hang] % [⟨shape⟩]
  {\normalfont\Large\bfseries\filright\sffamily} % {⟨format⟩}
  {\thesection} % {⟨label⟩}
  {2mm} % {⟨sep⟩}
  {%
    {#1\hspace{1mm}\color{colorAccent}{\Cdotfill}}%
  } % {⟨before-code⟩}

% NUMBERLESS - SUBSECTION
\titleformat{name=\subsection,numberless} % {⟨command⟩}
  [hang] % [⟨shape⟩]
  {\normalfont\Large\bfseries\filright\sffamily} % {⟨format⟩}
  {} % {⟨label⟩}
  {0mm} % {⟨sep⟩}
  {%
    {#1\hspace{1mm}\color{colorAccent}{\Cdotfill}}%
  }% {⟨before-code⟩}

\titlespacing*{\subsection} % <command>
  {0em} % <left>
  {.5em} % <before-sep>
  {.4em} % <after-sep>

% NUMBERED - SUBSUBSECTION
\titleformat{\subsubsection} % {⟨command⟩}
  [hang] % [⟨shape⟩]
  {\normalfont\large\bfseries\sffamily} % {⟨format⟩}
  {\thesection} % {⟨label⟩}
  {2mm} % {⟨sep⟩}
  {
    #1%
  } % {⟨before-code⟩}

% NUMBERLESS - SUBSUBSECTION
\titleformat{name=\subsubsection,numberless} % {⟨command⟩}
  [hang] % [⟨shape⟩]
  {\normalfont\large\bfseries\sffamily} % {⟨format⟩}
  {} % {⟨label⟩}
  {0mm} % {⟨sep⟩}
  {%
    #1%
  }% {⟨before-code⟩}

% [HEADER & FOOTER CONFIGURATION] ------------------------------------------------------ %
% Most configuration is done through the quarto project yaml file (aka. '_quarto.yml').
% 
% Inside '_quarto.yml' use following structure. If one is not used, uncomment it:
%
%  fancyhdr:
%    header:
%      right: "text"
%      center: "text"
%      left: "text"
%    footer:
%      right: "text"
%      center: "text"
%      left: "text"
\usepackage{fancyhdr}
\pagestyle{fancy}

% This adds the section number to the rightmark command, which is usually used for the
% header/footer
\renewcommand{\sectionmark}[1]{\markright{#1}}
\renewcommand{\subsectionmark}[1]{}

% Adds a rules to the header and footer
\renewcommand{\headrulewidth}{1pt}
\renewcommand{\footrulewidth}{1pt}

\fancyhf{} % clear all header and footer fields
% header
$if(fancyhdr.header.right)$
\fancyhead[R]{\sffamily $fancyhdr.header.right$}
$endif$

$if(fancyhdr.header.center)$
\fancyhead[C]{\sffamily $fancyhdr.header.center$}
$endif$

$if(fancyhdr.header.left)$
\fancyhead[L]{\sffamily $fancyhdr.header.left$}
$endif$

% header
$if(fancyhdr.footer.right)$
\fancyfoot[R]{\sffamily $fancyhdr.footer.right$}
$endif$

$if(fancyhdr.footer.center)$
\fancyfoot[C]{\sffamily $fancyhdr.footer.center$}
$endif$

$if(fancyhdr.footer.left)$
\fancyfoot[L]{\sffamily $fancyhdr.footer.left$}
$endif$

% [CONDITIONS ENVIRONMENT] ------------------------------------------------------------- %
% introduces conditions environment to create nice equation parameter description
% note: the first column is already in math mode, so no $$ are required.
%
% \begin{conditions}
%   A & Description about A \\
%   B & Description about B \\
%   C & Description about C
% \begin{conditions}
%
\usepackage{array}
\newenvironment{conditions}
  {\par\vspace{\abovedisplayskip}\noindent\begin{tabular}{>{$$}l<{$$} @{$${}:{}$$} l}}
  {\end{tabular}\par\vspace{\belowdisplayskip}}

% [FANCY VERBATIM] --------------------------------------------------------------------- %
% in combination with the file 'before-content.tex' inside the 'config' folder, the
% packages 'fancyvbr', 'fvextra' and 'floatrow' are used to restyle the implemented
% Verbatim/Codeblock styles to make it (personally) nicer looking and include breaklines +
% symbols
\usepackage{fancyvrb}
\definecolor{codeRuleColor}{HTML}{1a1e2e}
\definecolor{codeLineNumberColor}{HTML}{84858a}

% This reconfigures the code block to make a little nicer to look at.
% The following renewcommand 
\renewcommand{\theFancyVerbLine}{%
  \textcolor{codeLineNumberColor}{\ttfamily\scriptsize
  \arabic{FancyVerbLine}}}

\usepackage{fvextra}
\DefineVerbatimEnvironment{Highlighting} % Its new name
  {Verbatim} % Based on this environment
  {
    breaklines,
    breaksymbolleft={\textcolor{gray}{\scriptsize\ensuremath\hookrightarrow}},
    commandchars=\\\{\},
    rulecolor=\color{codeRuleColor},
    xleftmargin=1mm
  }


\DeclareFloatStyle{MyListingStyle}
  {
    style=plaintop,
    captionskip=1pt
  }

% [MISC] ------------------------------------------------------------------------------- %


% -------------------------------------------------------------------------------------- %

