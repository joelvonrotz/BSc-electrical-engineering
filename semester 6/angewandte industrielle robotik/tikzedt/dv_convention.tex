




\def\coordsys (#1,#2,#3) and (#4,#5,#6) node[#7] #8{
	% Set the shift coordinate
	\coordinate (shift) at (#1,#2,#3);
	% Set the origin of the rotated coordinate system
	\tdplotsetrotatedcoordsorigin{(shift)};
	
	% Apply the rotations
	\begin{scope}[tdplot_rotated_coords, rotate around x=#4]
    \begin{scope}[rotate around y=#5]
    \begin{scope}[rotate around z=#6]
        \draw[-Latex, thick, BrickRed]   (0,0,0) -- ++(1,0,0) node[pos=1.2, fill=white, inner sep=0.5] {$x$};
        \draw[-Latex, thick, OliveGreen] (0,0,0) -- ++(0,1,0) node[pos=1.2, fill=white, inner sep=0.5] {$y$};
        \draw[-Latex, thick, NavyBlue]   (0,0,0) -- ++(0,0,1) node[pos=1.2, fill=white, inner sep=0.5] {$z$};
        \node[circle, fill, inner sep=2] at (0,0,0) {};
        \node[#7] at (0,0,0) {#8};
	\end{scope}
	\end{scope}
	\end{scope}
}

\tdplotsetmaincoords{60}{-75}
\begin{tikzpicture}[tdplot_main_coords]
\draw[densely dashed, thick] (0,0,0) -- ++(0,0,2) -- ++(0,-2,0) -- ++(0,0,4);
\node[circle, fill, inner sep=2] at (0,0,0) {};

\coordsys (0, 0,2) and (0, 0, -90) node[left] {$K_0$};
\coordsys (0,-2,2) and (0,90, -90) node[left=8,below=-1] {$K_1$};
\coordsys (0,-2,6) and (0,90, -180) node[left] {$K_2$};
\end{tikzpicture}