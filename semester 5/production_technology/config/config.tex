\usepackage{amssymb, amsmath} % includes some neato math environments, symbols and stuff

\usepackage[utf8]{inputenc} % correctly encodes the formatting of the text files (since we live in the future of UTF8)
\usepackage{lastpage} % counts all the pages, used in footer

% Flexible Multicolumn Option
\usepackage{multicol}
\setlength\columnsep{20pt}

% Package for enabling colors (colorful output)
\usepackage[dvipsnames]{xcolor}
\definecolor{darkgreen}{HTML}{014f32}

% Icons - http://mirrors.ctan.org/fonts/fontawesome5/doc/fontawesome5.pdf
% This one is neat, when you want to add some icons
\usepackage{fontawesome5}

% Depending on the type of document, mostly summaries, I remove the spacing Display-styled Equations insert, since it's a waste of space.
\usepackage[nodisplayskipstretch]{setspace}

% [FONT CONFIGURATION]
\usepackage{lmodern}
\usepackage[defaultsans,scale=0.95]{opensans}
\usepackage[cm]{sfmath}
\usepackage[scaled=0.85]{beramono}
\renewcommand\familydefault{\sfdefault}


\usepackage{sectsty}
\allsectionsfont{\normalfont\sffamily\bfseries}

% [HEADING CUSTOMIZATION]
\usepackage[explicit]{titlesec}

% For numbered section creates:
%
%                         /|
% Heading                  |
% --------------------------
\titleformat
  {\section} % {⟨command⟩}
  [hang] % [⟨shape⟩]
  {\normalfont\LARGE\bfseries\filleft} % {⟨format⟩}
  {} % {⟨label⟩}
  {0mm} % {⟨sep⟩}
  {% {⟨before-code⟩}
    \parbox[b]{\dimexpr\linewidth-2.0cm\relax}{#1}\hfill%
    \parbox[b]{1.5cm}{\hfill{\fontsize{30}{36}\selectfont\thesection}}\\[-5mm]%
    \rule{\linewidth}{1pt}%
  }

% For unnumbered section creates:
%
% Heading
% --------------------------
\titleformat
  {name=\section,numberless} % {⟨command⟩}
  [hang] % [⟨shape⟩]
  {\normalfont\LARGE\bfseries\filleft} % {⟨format⟩}
  {} % {⟨label⟩}
  {0mm} % {⟨sep⟩}
  {% {⟨before-code⟩}
    \parbox[b]{\linewidth}{#1}\hfill\\[-5mm]%
    \rule{\linewidth}{1pt}%
  }


% [HEADER & FOOTER CONFIGURATION]
% ┌head───────────┐
% │ L     C     R │
% │ ------------- │ <- \headrulewidth
% │               │
% │               │
% │               │
% │               │
% │               │
% │ ------------- │ <- \footrulewidth
% │ L     C     R │
% └foot───────────┘
\usepackage{fancyhdr}

\renewcommand{\headrulewidth}{1pt}
\renewcommand{\footrulewidth}{1pt}

\pagestyle{fancy}
\fancyhead{} % clear all header fields
\fancyhead[R]{\sffamily Change this in \texttt{config/config.tex}}
\fancyhead[L]{\sffamily Change this in \texttt{config/config.tex}}

\fancyfoot{} % clear all footer fields
\fancyfoot[R]{\sffamily Joel von Rotz}
\fancyfoot[C]{\sffamily \thepage\ / \pageref{LastPage}}
\fancyfoot[L]{\sffamily \today}

% [CONDITIONS ENVIRONMENT]
% introduces conditions environment to create nice equation parameter description
% note: the first column is already in math mode, so no $ are required
%
% \begin{conditions}
%   A & Description about A \\
%   B & Description about B \\
%   C & Description about C
% \begin{conditions}
%
\usepackage{array}
\newenvironment{conditions}
  {\par\vspace{\abovedisplayskip}\noindent\begin{tabular}{>{$}l<{$} @{${}:{}$} l}}
  {\end{tabular}\par\vspace{\belowdisplayskip}}

% [TIKZ CONFIGURATION]
% Is used for different types of diagramms, drawings, etc.
\usepackage{tikz}
\newcommand*\circled[1]{\tikz[baseline=(char.base)]{
            \node[shape=circle,draw,inner sep=2pt] (char) {#1};}}
\usetikzlibrary{shapes,arrows,arrows.meta,matrix}

% [FANCYVRB]
\usepackage{fancyvrb}
\definecolor{codeRuleColor}{HTML}{1a1e2e}
\definecolor{codeLineNumberColor}{HTML}{84858a}

% This reconfigures the code block to make a little nicer to look at.
% The following renewcommand 
\renewcommand{\theFancyVerbLine}{%
  \textcolor{codeLineNumberColor}{\ttfamily\scriptsize
  \arabic{FancyVerbLine}}}

\DefineVerbatimEnvironment{Highlighting} % Its new name
  {Verbatim} % Based on this environment
  {
    commandchars=\\\{\},
    numbersep=4mm,
    rulecolor=\color{codeRuleColor},
    xleftmargin=4mm
  }


\usepackage{floatrow}
\DeclareFloatStyle{MyListingStyle}
  {
    style=plaintop,
    captionskip=1pt
  }
