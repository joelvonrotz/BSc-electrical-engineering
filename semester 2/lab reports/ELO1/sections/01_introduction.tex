% !TEX root = ../main.tex
\documentclass[../main.tex]{subfiles}

\begin{document}
\section{Einführung}



Dieser Messbericht und den dahinterliegenden Messversuch befasst sich mit den verschiedenen Verwendungsfällen eines \textit{bipolar junction transistor}'s (BJT).

Für die Messschaltung folgender Messungen wurden fixe Bezeichnungen für das Powersupply, Funktionsgenerator und Messgeräte genommen, welche in der folgenden Liste ersichtlich ist. Diese sollte als Referenz zu den Messschaltungen verwendet werden.

\begin{itemize}
    \item \textbf{PS/PS1/PS2} : RND 320-KD3305P Kanal 1, resp. Kanal 2
    \item \textbf{FL177-1} : Fluke 177 ET 14-099
    \item \textbf{FL177-2} : Fluke 177 ET 14-103
    \item \textbf{KO} : Agilent MSO6052A No.88
    \item \textbf{FG} : Funktionsgenerator jadadada
\end{itemize}
\end{document}
